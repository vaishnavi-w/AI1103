\documentclass[journal,12pt,twocolumn]{IEEEtran}

\usepackage{setspace}
\usepackage{gensymb}
\singlespacing
\usepackage[cmex10]{amsmath}

\usepackage{amsthm}

\usepackage{mathrsfs}
\usepackage{txfonts}
\usepackage{stfloats}
\usepackage{float}
\usepackage{bm}
\usepackage{cite}
\usepackage{cases}
\usepackage{subfig}

\usepackage{longtable}
\usepackage{multirow}

\usepackage{enumitem}
\usepackage{mathtools}
\usepackage{steinmetz}
\usepackage{tikz}
\usepackage{circuitikz}
\usepackage{verbatim}
\usepackage{tfrupee}
\usepackage[breaklinks=true]{hyperref}
\usepackage{graphicx}
\usepackage{tkz-euclide}

\usetikzlibrary{calc,math}
\usepackage{listings}
    \usepackage{color}                                            %%
    \usepackage{array}                                            %%
    \usepackage{longtable}                                        %%
    \usepackage{calc}                                             %%
    \usepackage{multirow}                                         %%
    \usepackage{hhline}                                           %%
    \usepackage{ifthen}                                           %%
    \usepackage{lscape}     
\usepackage{multicol}
\usepackage{chngcntr}

\DeclareMathOperator*{\Res}{Res}

\renewcommand\thesection{\arabic{section}}
\renewcommand\thesubsection{\thesection.\arabic{subsection}}
\renewcommand\thesubsubsection{\thesubsection.\arabic{subsubsection}}

\renewcommand\thesectiondis{\arabic{section}}
\renewcommand\thesubsectiondis{\thesectiondis.\arabic{subsection}}
\renewcommand\thesubsubsectiondis{\thesubsectiondis.\arabic{subsubsection}}


\hyphenation{op-tical net-works semi-conduc-tor}
\def\inputGnumericTable{}                                 %%

\lstset{
%language=C,
frame=single, 
breaklines=true,
columns=fullflexible
}
\begin{document}

\newcommand{\BEQA}{\begin{eqnarray}}
\newcommand{\EEQA}{\end{eqnarray}}
\newcommand{\define}{\stackrel{\triangle}{=}}
\bibliographystyle{IEEEtran}
\raggedbottom
\setlength{\parindent}{0pt}
\providecommand{\mbf}{\mathbf}
\providecommand{\pr}[1]{\ensuremath{\Pr\left(#1\right)}}
\providecommand{\qfunc}[1]{\ensuremath{Q\left(#1\right)}}
\providecommand{\sbrak}[1]{\ensuremath{{}\left[#1\right]}}
\providecommand{\lsbrak}[1]{\ensuremath{{}\left[#1\right.}}
\providecommand{\rsbrak}[1]{\ensuremath{{}\left.#1\right]}}
\providecommand{\brak}[1]{\ensuremath{\left(#1\right)}}
\providecommand{\lbrak}[1]{\ensuremath{\left(#1\right.}}
\providecommand{\rbrak}[1]{\ensuremath{\left.#1\right)}}
\providecommand{\cbrak}[1]{\ensuremath{\left\{#1\right\}}}
\providecommand{\lcbrak}[1]{\ensuremath{\left\{#1\right.}}
\providecommand{\rcbrak}[1]{\ensuremath{\left.#1\right\}}}
\newcommand\comb[2]{{}^{#1}\mathrm{C}_{#2}}
\theoremstyle{remark}
\newtheorem{rem}{Remark}
\newcommand{\sgn}{\mathop{\mathrm{sgn}}}
\providecommand{\abs}[1]{\vert#1\vert}
\providecommand{\res}[1]{\Res\displaylimits_{#1}} 
\providecommand{\norm}[1]{\lVert#1\rVert}
%\providecommand{\norm}[1]{\lVert#1\rVert}
\providecommand{\mtx}[1]{\mathbf{#1}}
\providecommand{\mean}[1]{E[ #1 ]}
\providecommand{\fourier}{\overset{\mathcal{F}}{ \rightleftharpoons}}
%\providecommand{\hilbert}{\overset{\mathcal{H}}{ \rightleftharpoons}}
\providecommand{\system}{\overset{\mathcal{H}}{ \longleftrightarrow}}
	%\newcommand{\solution}[2]{\textbf{Solution:}{#1}}
\newcommand{\solution}{\noindent \textbf{Solution: }}
\newcommand{\cosec}{\,\text{cosec}\,}
\providecommand{\dec}[2]{\ensuremath{\overset{#1}{\underset{#2}{\gtrless}}}}
\newcommand{\myvec}[1]{\ensuremath{\begin{pmatrix}#1\end{pmatrix}}}
\newcommand{\mydet}[1]{\ensuremath{\begin{vmatrix}#1\end{vmatrix}}}
\numberwithin{equation}{subsection}
\makeatletter
\@addtoreset{figure}{problem}
\makeatother
\let\StandardTheFigure\thefigure
\let\vec\mathbf
\renewcommand{\thefigure}{\theproblem}
\def\putbox#1#2#3{\makebox[0in][l]{\makebox[#1][l]{}\raisebox{\baselineskip}[0in][0in]{\raisebox{#2}[0in][0in]{#3}}}}
     \def\rightbox#1{\makebox[0in][r]{#1}}
     \def\centbox#1{\makebox[0in]{#1}}
     \def\topbox#1{\raisebox{-\baselineskip}[0in][0in]{#1}}
     \def\midbox#1{\raisebox{-0.5\baselineskip}[0in][0in]{#1}}
\vspace{3cm}
\title{AI1103-Assignment 6}
\author{W Vaishnavi\\AI20BTECH11025}
\maketitle
\newpage
\bigskip
\renewcommand{\thefigure}{\theenumi}
\renewcommand{\thetable}{\theenumi}
Download all latex-tikz codes from 
%
\begin{lstlisting}
https://github.com/vaishnavi-w/AI1103/blob/main/Assignment6/latex6.tex
\end{lstlisting}
\section*{Question}
Which of the following conditions imply independence of random variables $X$ and $Y$?
\begin{enumerate}
    \item $\pr{X>a|Y>a}=\pr{X>a} $ for all $a\in R$
    \item $\pr{X>a|Y<b}=\pr{X>a} $ for all $a,b\in R$
    \item $X$ and $Y$ are uncorrelated
    \item $E\sbrak{\brak{X-a}\brak{Y-b}}=E\sbrak{X-a}E\sbrak{Y-b}$ for all $a,b\in R$
\end{enumerate} 
\section*{Solution}
\begin{enumerate}
    \item Two random variables $X$ and $Y$ are independent when the joint probability of random variables is product of their individual probabilities
    \begin{align}
        \label{eq1} \pr{X\in A,Y \in B}=\pr{X \in A}\pr{Y \in B}
    \end{align}
    In other words conditional probability follows
    \begin{align}
        \pr{X\in A|Y \in B}=\pr{X \in A}
    \end{align}
    for all sets $A$ and $B$. \\
    In the case of $\pr{X>a|Y>a}=\pr{X>a}$ the sets $A=B$. The spectrum of conditions for independence is underrepresented. Hence, the condition does not imply independence of $X$ and $Y$.\\
    \textbf{Counterexample:} Consider two random variables $X$,$Y \in \{0,1,2\}$ with the probabilities of the ordered pairs \brak{X,Y} given in the Table\ref{table1}
    \begin{center}
    \begin{table}[h]
    \centering
    \resizebox{\columnwidth}{!}{
    \begin{tabular}{|c|c||c|c||c|c|}
    \hline
    \brak{X,Y}&Pr&\brak{X,Y}&Pr&\brak{X,Y}&Pr\\ 
    \hline
    (0,0)&0.2& (1,0)&0.2& (2,0)&0.1 \\ 
    (0,1)&0.1& (1,1)&0.1& (2,1)&0.1 \\
    (0,2)&0.1& (1,2)&0.05& (2,2)&0.05 \\
    \hline
    \end{tabular}
    }
    \caption{\pr{X,Y}}
    \label{table1}
    \end{table}
    \end{center}
    
    Consider
    \begin{multline}
        \pr{X>1|Y>1}=\frac{\pr{X>1,Y>1}}{\pr{Y>1}}\\
        =\frac{\pr{X=2,Y=2}}{\pr{Y=2}}=\frac{0.05}{0.2}=0.25
    \end{multline}
    \begin{align}
        \pr{X>1}=\pr{X=2}=0.25
    \end{align}
    Similarly, we can find that 
    $\pr{X>a|Y>a}=\pr{X>a}$ for all $a \leq 2 $.
    ($\because$ $\pr{Y>a}=0$ for all $a>2$)\\
    Consider,
    \begin{align}
        \pr{X=1,Y=2}=0.05
    \end{align}
    \begin{multline}
        \pr{X=1}\pr{Y=2}=0.35\times0.2=0.7\\\neq\pr{X=1,Y=2}
    \end{multline} 
    Clearly, $X$ and $Y$ are not independent. 
    
    \textbf{Option 1 is incorrect}
    
    \item From Bayes theorem,
\begin{align}
    \pr{X>a|Y<b}=\frac{\pr{\brak{X>a},\brak{Y<b}}}{\pr{Y<b}}
\end{align}
Given that $\pr{X>a|Y<b}=\pr{X>a}$,
\begin{align}
    \frac{\pr{\brak{X>a},\brak{Y<b}}}{\pr{Y<b}}=\pr{X>a}\\
    \pr{\brak{X>a},\brak{Y<b}}=\pr{X>a}\pr{Y<b}
\end{align}
for all $a,b\in R$.\\
From \eqref{eq1}, $X$ and $Y$ are independent. \\
\textbf{Option 2 is correct}
    \item Two random variables $X$ and $Y$ are uncorrelated if their covariance is zero.
\begin{align}
    cov\sbrak{X,Y}=E\sbrak{XY}-E\sbrak{X}E\sbrak{Y}=0
\end{align}
    Uncorrelatedness does not imply independence.\\
    \textbf{Counterexample:} Let $X\sim U \sbrak{-1,1}$ be a uniformly distributed random variable.
    \begin{align}
        f_X\brak{x}=
        \begin{cases}
        \frac{1}{2} & -1\leq x \leq 1\\
        0 & otherwise
        \end{cases}\\
        E\sbrak{X}=\int_{-1}^{1}x f\brak{x} dx=0
    \end{align}
    Let $Y=X^2$ be another random variable.\\
    $X$ and $Y$ are dependent.
    \begin{align}
        cov\sbrak{X,Y}&=E\sbrak{XY}-E\sbrak{X}E\sbrak{Y}\\
        &=E\sbrak{X^3}-0\times E\sbrak{Y}\\
        &=\int_{-1}^{1}x^3 f\brak{x} dx=0
    \end{align}
    $X$ and $Y$ are uncorrelated but not independent.\\
    \textbf{Option 3 is incorrect}
    \item Given that,
    \begin{align}
        E\sbrak{\brak{X-a}\brak{Y-b}}=E\sbrak{X-a}E\sbrak{Y-b}
    \end{align}
    \begin{multline}
        cov\sbrak{\brak{X-a},\brak{Y-b}}=
        E\sbrak{\brak{X-a}\brak{Y-b}}\\-E\sbrak{X-a}E\sbrak{Y-b}
    \end{multline}
    \begin{align}
        \implies cov\sbrak{\brak{X-a}\brak{Y-b}}=0
    \end{align}
    From option 3, it follows that $X$ and $Y$ are not necessarily independent.\\
    \textbf{Option 4 is incorrect.}
\end{enumerate}
\end{document}
