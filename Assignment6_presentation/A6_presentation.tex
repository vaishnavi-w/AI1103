\documentclass{beamer}
\usepackage{listings}
\lstset{
%language=C,
frame=single, 
breaklines=true,
columns=fullflexible
}
\usepackage{blkarray}
\usepackage{subcaption}
\usepackage{url}
\usepackage{tikz}
\usepackage{tkz-euclide} % loads  TikZ and tkz-base
%\usetkzobj{all}
\usetikzlibrary{calc,math}
\usepackage{float}
\newcommand\norm[1]{\left\lVert#1\right\rVert}
\renewcommand{\vec}[1]{\mathbf{#1}}
\usepackage[export]{adjustbox}
\usepackage[utf8]{inputenc}
\usepackage{amsmath}
\usepackage{tikz}
\usepackage{diagbox}
\usetikzlibrary{automata, positioning}
\usetheme{Boadilla}
\providecommand{\pr}[1]{\ensuremath{\Pr\left(#1\right)}}
\providecommand{\brak}[1]{\ensuremath{\left(#1\right)}}
\providecommand{\sbrak}[1]{\ensuremath{{}\left[#1\right]}}
\DeclareMathOperator*{\max}{max}
\title{CSIR UGC NET EXAM (June 2012), Q.104}
\author{Vaishnavi W}
\date{AI20BTECH11025}
\begin{document}

\begin{frame}
\titlepage
\end{frame}

\begin{frame}
\begin{block}{Independence of Random Variables}
    Two random variables $X$ and $Y$ are independent when the joint probability distribution of random variables is product of their individual probability distributions i.e for all sets A,B
    \begin{align}
        \label{eq1} \pr{X\in A,Y \in B}=\pr{X \in A}\pr{Y \in B}
    \end{align}
\end{block}
    
\end{frame}
\begin{frame}{Independence of Random Variables}
    Consider the CDFs,
    \begin{align}
        F_X\brak{a}=\pr{X<a}\\
        F_Y\brak{b}=\pr{Y<b}\\
        F_{X,Y}\brak{a,b}=\pr{X<a,Y<b}
    \end{align}
    Let $F_{X,Y}\brak{a,b}=F_X\brak{a}F_Y\brak{b}$ be true. 
    
    Partial derivative w.r.t a and then w.r.t b,
    \begin{align}
        \frac{\partial^2 F_{X,Y}\brak{a,b}}{\partial b \partial a}=\frac{\partial F_X\brak{a} }{\partial a}\frac{\partial F_Y\brak{b} }{\partial b} \\
        \implies p_{X,Y}\brak{a,b}=p_X\brak{a}p_Y\brak{b}
    \end{align}
    when X,Y are discrete. And,
    \begin{align}
        \implies f_{X,Y}\brak{a,b}=f_X\brak{a}f_Y\brak{b}
    \end{align}
    when X,Y are continuous, for all $a,b\in R$. 
\end{frame}
\begin{frame}
\begin{block}{Independence of Random Variables}
    Two random variables are independent if the joint CDF can be expressed as the product of individual CDFs i.e for all $a,b\in R$
    \begin{align}
       \label{eq1} F_{X,Y}\brak{a,b}=F_X\brak{a}F_Y\brak{b}
    \end{align}
\end{block}
\end{frame}
\begin{frame}{Question}
\begin{block}{CSIR UGC NET EXAM (June 2012), Q.104}
Which of the following conditions imply independence of random variables $X$ and $Y$?
\begin{enumerate}
    \item $\pr{X>a|Y>a}=\pr{X>a} $ for all $a\in R$
    \item $\pr{X>a|Y<b}=\pr{X>a} $ for all $a,b\in R$
    \item $X$ and $Y$ are uncorrelated
    \item $E\sbrak{\brak{X-a}\brak{Y-b}}=E\sbrak{X-a}E\sbrak{Y-b}$ for all $a,b\in R$
\end{enumerate} 
\end{block}
\end{frame}
\begin{frame}{Option 1}
    Consider,
    \begin{align}
      \pr{X>a|Y>a}&=\frac{\pr{X>a,Y>a}}{\pr{Y>a}}
    \end{align}
    Given that $\pr{X>a|Y>a}=\pr{X>a}$,
    \begin{align}
        \label{eq2}\implies \pr{X>a,Y>a}&=\pr{X>a}\pr{Y>a}
    \end{align}
    for all $a\in R$
    
\end{frame}
\begin{frame}{Solution contd.}
    \begin{multline}
        1-F_X\brak{a}-F_Y\brak{a}=\pr{X>a}-\pr{Y<a}\\=\pr{X>a,Y>a}+\pr{X>a,Y<a}-\pr{X>a,Y<a}\\ -\pr{X<a,Y<a}
    \end{multline}
    \begin{align}
        1-F_X\brak{a}-F_Y\brak{a}=\pr{X>a,Y>a}-F_{X,Y}\brak{a,a}
    \end{align}
    Substituting in \eqref{eq2},
    \begin{align}
        1-F_X\brak{a}-F_Y\brak{a}+F_{X,Y}\brak{a,a}=\brak{1-F_X\brak{a}}\brak{1-F_Y\brak{a}}\\
        \implies F_{X,Y}\brak{a,a}=F_X\brak{a}F_Y\brak{a}
    \end{align}  
    for all $a\in R$
\end{frame}
\begin{frame}{Solution contd.}
 On comparing with \eqref{eq1} in this case, it is true only under the condition that $b=a$. It may not be true for other values of $b$. The spectrum of conditions for independence is underrepresented. Hence, the condition does not imply independence of $X$ and $Y$.
  
  \textbf{Counterexample:} Consider two random variables $X$,$Y \in \{0,1,2\}$ with the probabilities of the ordered pairs \brak{X,Y} given in the Table\ref{table1}
  \begin{center}
    \begin{table}[H]
    
    \centering
    
    \begin{tabular}{|l|c|c|c|}
    \hline
    \backslashbox{$X$}{$Y$}&0&1&2\\ 
    \hline
     0&0.2&0.1 &0.1\\
     1&0.2&0.1&0.05\\
     2&0.1&0.1&0.05\\
    \hline
    \end{tabular}
    \caption{\pr{X,Y}}
    \label{table1}
    
    \end{table}
    \end{center}
\end{frame}
\begin{frame}{Solution contd.}
    In all the cases, $\pr{X>a|Y>a}=\pr{X>a}$ is true.
    
    Consider,
    \begin{align}
        \pr{X=1,Y=2}=0.05\\
        \pr{X=1}\pr{Y=2}=0.35\times0.2=0.7\neq\pr{X=1,Y=2}
    \end{align} 
    Clearly, $X$ and $Y$ are not independent. 
\end{frame}
\begin{frame}{Option 2 }
    \begin{align}
      \pr{X>a|Y<b}=\frac{\pr{X>a,Y<b}}{\pr{Y<b}}
    \end{align}
    Given that $\pr{X>a|Y<b}=\pr{X>a}$,
    \begin{align}
        \label{eq3}\implies \pr{X>a,Y<b}=\pr{X>a}\pr{Y<b}
    \end{align}
    for all $a,b\in R$. Consider
    \begin{align}
        F_Y\brak{b}=\pr{X>a,Y<b}+\pr{X<a,Y<b}\\
        \implies F_Y\brak{b}-F_{X,Y}\brak{a,b}=\pr{X>a,Y<b}
    \end{align}
    Substituting in \eqref{eq3},
    \begin{align}
        F_Y\brak{b}-F_{X,Y}\brak{a,b}=\brak{1-F_X\brak{a}}F_Y\brak{b}\\
        \implies F_{X,Y}\brak{a,b}=F_X\brak{a}F_Y\brak{b}
    \end{align}
    for all $a,b\in R$. Thus, $X$ and $Y$ are independent. 
\end{frame}
\begin{frame}{Option 3}
\begin{block}{Uncorrelatedness}
    Two random variables $X$ and $Y$ are uncorrelated if their covariance is zero.
    \begin{align}
        cov\sbrak{X,Y}=E\sbrak{XY}-E\sbrak{X}E\sbrak{Y}=0
    \end{align}
    Uncorrelatedness does not imply independence.
\end{block}
\end{frame}
\begin{frame}{Solution contd.}
    \textbf{Counterexample:} Let $X\sim U \sbrak{-1,1}$ be a uniformly distributed random variable.
    \begin{align}
        f_X\brak{x}=
        \begin{cases}
        \frac{1}{2} & -1\leq x \leq 1\\
        0 & otherwise
        \end{cases}\\
        E\sbrak{X}=\int_{-1}^{1}x f\brak{x} dx=0
    \end{align}
    Let $Y=X^2$ be another random variable.
    $X$ and $Y$ are dependent.
    \begin{align}
        cov\sbrak{X,Y}&=E\sbrak{XY}-E\sbrak{X}E\sbrak{Y}\\
        &=E\sbrak{X^3}-0\times E\sbrak{Y}\\
        &=\int_{-1}^{1}x^3 f\brak{x} dx=0
    \end{align}
    $X$ and $Y$ are uncorrelated but not independent.
\end{frame}
\begin{frame}{Option 4}
    Given that,
    \begin{align}
        E\sbrak{\brak{X-a}\brak{Y-b}}=E\sbrak{X-a}E\sbrak{Y-b}\\
        E\sbrak{XY-aY-bX+ab}=\brak{E\sbrak{X}-a}\brak{E\sbrak{Y}-b}\\
        E\sbrak{XY}-aE\sbrak{Y}-bE\sbrak{X}+ab = \brak{E\sbrak{X}-a}\brak{E\sbrak{Y}-b}\\
        \implies E\sbrak{XY} = E\sbrak{X}E\sbrak{Y}
    \end{align}
    \begin{align}
        cov\sbrak{X,Y}&=E\sbrak{XY}-E\sbrak{X}E\sbrak{Y}=0
    \end{align}
    From option 3, it follows that $X$ and $Y$ are not necessarily independent.
\end{frame}
\end{document}