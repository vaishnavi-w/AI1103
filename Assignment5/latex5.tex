\documentclass[journal,12pt,twocolumn]{IEEEtran}

\usepackage{setspace}
\usepackage{gensymb}
\singlespacing
\usepackage[cmex10]{amsmath}

\usepackage{amsthm}

\usepackage{mathrsfs}
\usepackage{txfonts}
\usepackage{stfloats}

\usepackage{bm}
\usepackage{cite}
\usepackage{cases}
\usepackage{subfig}

\usepackage{longtable}
\usepackage{multirow}

\usepackage{enumitem}
\usepackage{mathtools}
\usepackage{steinmetz}
\usepackage{tikz}
\usepackage{circuitikz}
\usepackage{verbatim}
\usepackage{tfrupee}
\usepackage[breaklinks=true]{hyperref}
\usepackage{graphicx}
\usepackage{tkz-euclide}

\usetikzlibrary{calc,math}
\usepackage{listings}
    \usepackage{color}                                            %%
    \usepackage{array}                                            %%
    \usepackage{longtable}                                        %%
    \usepackage{calc}                                             %%
    \usepackage{multirow}                                         %%
    \usepackage{hhline}                                           %%
    \usepackage{ifthen}                                           %%
    \usepackage{lscape}     
\usepackage{multicol}
\usepackage{chngcntr}

\DeclareMathOperator*{\Res}{Res}

\renewcommand\thesection{\arabic{section}}
\renewcommand\thesubsection{\thesection.\arabic{subsection}}
\renewcommand\thesubsubsection{\thesubsection.\arabic{subsubsection}}

\renewcommand\thesectiondis{\arabic{section}}
\renewcommand\thesubsectiondis{\thesectiondis.\arabic{subsection}}
\renewcommand\thesubsubsectiondis{\thesubsectiondis.\arabic{subsubsection}}


\hyphenation{op-tical net-works semi-conduc-tor}
\def\inputGnumericTable{}                                 %%

\lstset{
%language=C,
frame=single, 
breaklines=true,
columns=fullflexible
}
\begin{document}

\newcommand{\BEQA}{\begin{eqnarray}}
\newcommand{\EEQA}{\end{eqnarray}}
\newcommand{\define}{\stackrel{\triangle}{=}}
\bibliographystyle{IEEEtran}
\raggedbottom
\setlength{\parindent}{0pt}
\providecommand{\mbf}{\mathbf}
\providecommand{\pr}[1]{\ensuremath{\Pr\left(#1\right)}}
\providecommand{\qfunc}[1]{\ensuremath{Q\left(#1\right)}}
\providecommand{\sbrak}[1]{\ensuremath{{}\left[#1\right]}}
\providecommand{\lsbrak}[1]{\ensuremath{{}\left[#1\right.}}
\providecommand{\rsbrak}[1]{\ensuremath{{}\left.#1\right]}}
\providecommand{\brak}[1]{\ensuremath{\left(#1\right)}}
\providecommand{\lbrak}[1]{\ensuremath{\left(#1\right.}}
\providecommand{\rbrak}[1]{\ensuremath{\left.#1\right)}}
\providecommand{\cbrak}[1]{\ensuremath{\left\{#1\right\}}}
\providecommand{\lcbrak}[1]{\ensuremath{\left\{#1\right.}}
\providecommand{\rcbrak}[1]{\ensuremath{\left.#1\right\}}}
\theoremstyle{remark}
\newtheorem{rem}{Remark}
\newcommand{\sgn}{\mathop{\mathrm{sgn}}}
\providecommand{\abs}[1]{\vert#1\vert}
\providecommand{\res}[1]{\Res\displaylimits_{#1}} 
\providecommand{\norm}[1]{\lVert#1\rVert}
%\providecommand{\norm}[1]{\lVert#1\rVert}
\providecommand{\mtx}[1]{\mathbf{#1}}
\providecommand{\mean}[1]{E[ #1 ]}
\providecommand{\fourier}{\overset{\mathcal{F}}{ \rightleftharpoons}}
%\providecommand{\hilbert}{\overset{\mathcal{H}}{ \rightleftharpoons}}
\providecommand{\system}{\overset{\mathcal{H}}{ \longleftrightarrow}}
	%\newcommand{\solution}[2]{\textbf{Solution:}{#1}}
\newcommand{\solution}{\noindent \textbf{Solution: }}
\newcommand{\cosec}{\,\text{cosec}\,}
\providecommand{\dec}[2]{\ensuremath{\overset{#1}{\underset{#2}{\gtrless}}}}
\newcommand{\myvec}[1]{\ensuremath{\begin{pmatrix}#1\end{pmatrix}}}
\newcommand{\mydet}[1]{\ensuremath{\begin{vmatrix}#1\end{vmatrix}}}
\numberwithin{equation}{subsection}
\makeatletter
\@addtoreset{figure}{problem}
\makeatother
\let\StandardTheFigure\thefigure
\let\vec\mathbf
\renewcommand{\thefigure}{\theproblem}
\def\putbox#1#2#3{\makebox[0in][l]{\makebox[#1][l]{}\raisebox{\baselineskip}[0in][0in]{\raisebox{#2}[0in][0in]{#3}}}}
     \def\rightbox#1{\makebox[0in][r]{#1}}
     \def\centbox#1{\makebox[0in]{#1}}
     \def\topbox#1{\raisebox{-\baselineskip}[0in][0in]{#1}}
     \def\midbox#1{\raisebox{-0.5\baselineskip}[0in][0in]{#1}}
\vspace{3cm}
\title{AI1103-Assignment 5}
\author{W Vaishnavi\\AI20BTECH11025}
\maketitle
\newpage
\bigskip
\renewcommand{\thefigure}{\theenumi}
\renewcommand{\thetable}{\theenumi}
Download all latex-tikz codes from 
%
\begin{lstlisting}
https://github.com/vaishnavi-w/AI1103/blob/main/Assignment5/latex5.tex
\end{lstlisting}
\section*{Question}
Let $X_1$ and $X_2$ be i.i.d. with probability mass function $f_{\theta}\brak{x} = \theta^x \brak{1-\theta}^{1-x}$; $x=0,1$ where $\theta \in \brak{0,1}$. Which of the following statements are true?
\begin{enumerate}
    \item $X_1 + 2X_2 $ is a sufficient statistic
    \item $X_1 - X_2 $ is a sufficient statistic
    \item $X_1^2 + X_2^2 $ is a sufficient statistic
    \item $X_1^2 + X_2 $ is a sufficient statistic
\end{enumerate}
\section*{Solution}
A statistic $t=T\brak{X}$ is sufficient for a parameter $\theta$ if the conditional probability distribution of the data, given the statistic $t=T\brak{X}$ does not depend on the parameter $\theta$. i.e,
\begin{align}
    P_\theta\brak{X_1=x_1,X_2=x_2|T=t}
\end{align}
is independent of $\theta$ for all $x_1,x_2$ and $t$
\begin{enumerate}
    \item Let $T=X_1+2X_2$
    \begin{align}
    \pr{T=0}&=\pr{X_1+2X_2=0}\\
    &=\pr{X_1=0,X_2=0}
    \end{align}
    As $X_1$ and $X_2$ are independent
    \begin{multline}
        \pr{T=0}=\pr{X_1=0}\pr{X_2=0}\\
    =\theta^0 \brak{1-\theta}^{1-0}\times\theta^0 \brak{1-\theta}^{1-0}=\brak{1-\theta}^2
    \end{multline}
    Consider,
    \begin{multline}
        \pr{X_1=0,X_2=0 | T=0}\\=\frac{\pr{X_1=0,X_2=0}}{\pr{T=0}}
        =\frac{\brak{1-\theta}^2}{\brak{1-\theta}^2}=1
    \end{multline}
    Similarly, conditional probabilities for other values of $x_1,x_2$ and $t$ are given in table \ref{table1}
    \begin{table}[h!]
    \begin{tabular}[width=\columnwidth]{|c|c|c|c|}
         \hline
        \multirow{2}{*}{\textbf{x_1}} & \multirow{2}{*}{\textbf{x_2}} & \textbf{t} & \textbf{Conditional probability}  \\
        & & $t=X_1+2X_2$ & $P_\theta\brak{X_1=x_1,X_2=x_2|T=t}$\\
        \hline
        \multirow{2}{*}{0} & \multirow{2}{*}{0} & 0 & 1\\ 
        & & otherwise & 0 \\ 
        \hline
        \multirow{2}{*}{1} & \multirow{2}{*}{0} & 1 & 1\\ 
        & & otherwise & 0 \\ 
        \hline
        \multirow{2}{*}{0} & \multirow{2}{*}{1} & 2 & 1\\ 
        & & otherwise & 0 \\ 
        \hline
        \multirow{2}{*}{1} & \multirow{2}{*}{1} & 3 & 1\\ 
        & & otherwise & 0 \\        
        \hline
    \end{tabular}
    \caption{Conditional Probabilities}
    \label{table1}
    \end{table}    
    
    From table \ref{table1}, all the conditional probabilities are independent of $\theta$\\ $\therefore X_1+2X_2$ is a sufficient statistic.
    
    
    \item Let $T=X_1-X_2$
    \begin{multline}
        \pr{T=0}=\pr{X_1-X_2=0}\\=\pr{X_1=0,X_2=0}+\pr{X_1=1,X_2=1}
    \end{multline}
    As $X_1$ and $X_2$ are independent
    \begin{multline}
        =\pr{X_1=0}\pr{X_2=0}\\+\pr{X_1=1}\pr{X_2=1}=\brak{1-\theta}^2 + \theta^2
    \end{multline}
    Consider,
    \begin{multline}
        \pr{X_1=0,X_2=0 | T=0} = \\\frac{\pr{X_1=0,X_2=0}}{\pr{T=0}}
        =\frac{\brak{1-\theta}^2}{\brak{1-\theta}^2 + \theta^2}
    \end{multline}
    depends on $\theta$. \\
    $\therefore X_1-X_2$ is not a sufficient statistic.
    
    \item Let $T=X_1^2+X_2^2$
    \begin{multline}
        \pr{T=1}=\pr{X_1^2+X_2^2=1}\\=\pr{X_1=1,X_2=0}+\pr{X_1=0,X_2=1}
    \end{multline}
     \begin{align}
        =\theta \brak{1-\theta} + \brak{1-\theta}\theta  = 2\theta \brak{1-\theta}
    \end{align}
    Consider,
    \begin{multline}
        \pr{X_1=1,X_2=0 | T=1} = \\\frac{\pr{X_1=1,X_2=0}}{\pr{T=1}}
        =\frac{\brak{1-\theta}}{2\brak{1-\theta}^2}=\frac{1}{2}
    \end{multline}
    Similarly, conditional probabilities for other values of $x_1,x_2$ and $t$ are given in table \ref{table2}
    \begin{table}[h!]
    \begin{tabular}[width=\columnwidth]{|c|c|c|c|}
         \hline
        \multirow{2}{*}{\textbf{x_1}} & \multirow{2}{*}{\textbf{x_2}} & \textbf{t} & \textbf{Conditional probability}  \\
        & & $t=X_1^2+X_2^2$ & $P_\theta\brak{X_1=x_1,X_2=x_2|T=t}$\\
        \hline
        \multirow{2}{*}{0} & \multirow{2}{*}{0} & 0 & 1\\ 
        & & otherwise & 0 \\ 
        \hline
        \multirow{2}{*}{1} & \multirow{2}{*}{0} & 1 & $\frac{1}{2}$\\ 
        & & otherwise & 0 \\ 
        \hline
        \multirow{2}{*}{0} & \multirow{2}{*}{1} & 1 & $\frac{1}{2}$\\ 
        & & otherwise & 0 \\ 
        \hline
        \multirow{2}{*}{1} & \multirow{2}{*}{1} & 2 & 1\\ 
        & & otherwise & 0 \\        
        \hline
    \end{tabular}
    \caption{Conditional Probabilities}
    \label{table2}
    \end{table}  
    
    From table \ref{table2}, all the conditional probabilities are independent of $\theta$\\ $\therefore X_1^2+X_2^2$ is a sufficient statistic.
    
    \item Let $T=X_1^2+X_2$\\
    \begin{multline}
        \pr{T=1}=\pr{X_1^2+X_2=1}\\=\pr{X_1=1,X_2=0}+\pr{X_1=0,X_2=1}
    \end{multline}
     \begin{align}
        =\theta \brak{1-\theta} + \brak{1-\theta}\theta  = 2\theta \brak{1-\theta}
    \end{align}
    Consider,
    \begin{multline}
        \pr{X_1=1,X_2=0 | T=1} = \\\frac{\pr{X_1=1,X_2=0}}{\pr{T=1}}
        =\frac{\brak{1-\theta}}{2\brak{1-\theta}^2}=\frac{1}{2}
    \end{multline}
    Similarly, conditional probabilities for other values of $x_1,x_2$ and $t$ are given in table \ref{table3}
    
    \begin{center}
    \begin{table}[H]
    \begin{tabular}[width=\columnwidth]{|c|c|c|c|}
         \hline
        \multirow{2}{*}{\textbf{x_1}} & \multirow{2}{*}{\textbf{x_2}} & \textbf{t} & \textbf{Conditional probability}  \\
        & & $t=X_1^2+X_2$ & $P_\theta\brak{X_1=x_1,X_2=x_2|T=t}$\\
        \hline
        \multirow{2}{*}{0} & \multirow{2}{*}{0} & 0 & 1\\ 
        & & otherwise & 0 \\ 
        \hline
        \multirow{2}{*}{1} & \multirow{2}{*}{0} & 1 & $\frac{1}{2}$\\ 
        & & otherwise & 0 \\ 
        \hline
        \multirow{2}{*}{0} & \multirow{2}{*}{1} & 1 & $\frac{1}{2}$\\ 
        & & otherwise & 0 \\ 
        \hline
        \multirow{2}{*}{1} & \multirow{2}{*}{1} & 2 & 1\\ 
        & & otherwise & 0 \\        
        \hline
    \end{tabular}
    \caption{Conditional Probabilities}
    \label{table3}
    \end{table}  
    \end{center}
    From table \ref{table3}, all the conditional probabilities are independent of $\theta$ $\therefore X_1^2+X_2$ is a sufficient statistic.
    
\end{enumerate}
\rightline{Answer : Options 1,3,4}
\end{document}
